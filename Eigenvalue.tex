\documentclass[12pt]{article}

% This first part of the file is called the PREAMBLE. It includes
% customizations and command definitions. The preamble is everything
% between \documentclass and \begin{document}.

\usepackage[margin=1in]{geometry}  % set the margins to 1in on all sides
\usepackage{graphicx}              % to include figures
\usepackage{amsmath}               % great math stuff
\usepackage{amsfonts}              % for blackboard bold, etc
\usepackage{amsthm}                % better theorem environments
\usepackage{amsthm,amssymb}


\usepackage{amsmath}
\usepackage{amsfonts}
\usepackage{amssymb}
\usepackage{amsxtra}
\usepackage{amsthm}
\usepackage{mathrsfs}
\usepackage{color}

%Here I define some theorem styles and shortcut commands for symbols I use often
\theoremstyle{definition}
\newtheorem{defn}{Definition}
\newtheorem{thm}{Theorem}
\newtheorem{cor}{Corollary}
\newtheorem*{rmk}{Remark}
\newtheorem{lem}{Lemma}
\newtheorem*{joke}{Joke}
\newtheorem{ex}{Example}
\newtheorem*{soln}{Solution}
\newtheorem{prop}{Proposition}

\DeclareMathOperator{\id}{id}

\newcommand{\bd}[1]{\mathbf{#1}}  % for bolding symbols
\newcommand{\RR}{\mathbb{R}}      % for Real numbers
\newcommand{\ZZ}{\mathbb{Z}}      % for Integers
\newcommand{\col}[1]{\left[\begin{matrix} #1 \end{matrix} \right]}
\newcommand{\comb}[2]{\binom{#1^2 + #2^2}{#1+#2}}

\begin{document}


\nocite{}

\title{\textbf{Eigenvalues and Diagonalization}}

\author{Miliyon T.}
\date{August 11, 2015}
\maketitle


\section*{Eigenvalues and Eigenvectors}

\begin{defn}
Let $A$ be an $n\times n$ matrix. The scalar $\lambda$ is called an eigenvalue\footnote{The terms eigenvalue and eigenvector are derived from the German word \textit{Eigenwert}(Proper value).} of $A$ if there is a nonzero vector $\mathbf{x}$ such that
\[A\mathbf{x}=\lambda \mathbf{x}\]
The vector $\mathbf{x}$ is called an eigenvector of $A$ corresponding to $\lambda$.
\end{defn}

\begin{rmk}
Note that an eigenvector cannot be zero. Allowing $\mathbf{x}$ to be the zero vector would render the definition meaningless, because $A\mathbf{0}=\lambda \mathbf{0}$ is true for all real values of $\lambda$.
An eigenvalue of $\lambda =0$, however, is possible.
\end{rmk}
\subsection*{Definition}

\begin{itemize}
\item Let $A=(a_{ij})$ be a square matrix of order n over F$(M_n(F))$. A vector $x\in F^n$ is called an \textbf{eigenvector} of $A$ if $ \exists \lambda \in F$ such that
      \[Ax=\lambda x\]
  Where $\lambda$-\textbf{eigenvalue} of $A$ corresponding to the \textbf{eigenvector} $x$.

\subitem Geometric multiplicity
\subitem Algebraic multiplicity
\item  $|A-\lambda I_n|=\det(A-\lambda I_n)$ is a polynomial of degree $n$ known as characteristics polynomial of $A$ denoted by $P(\lambda )=|A-\lambda I_n|$
\item  The equation $|\lambda I_n -A|=0$ is called \textbf{characteristic equation} of a matrix $A$.
\item \emph{\textbf{Similar Matrix}}: The matrices $A$ and $B\in M_n(F)$ are Similar if there exists an invertible matrix $P$ such that
   \[B=P^{-1}AP\]
\item \emph{\textbf{Diagonalizable matrix}}: Let $A\in M_n(F)$. $A$ is said to be diagonalizable if $A$ is similar to a diagonal matrix $D$.
    \[A=PDP^{-1}\]

\end{itemize}
\newpage
\begin{ex}
Find the eigenvalues and corresponding eigenvectors of
\[A=\begin{bmatrix}
2 & -4\\
3 & -5
\end{bmatrix}\]
\end{ex}
\begin{proof}[Solution]
The characteristic polynomial of $A$ is
\begin{align*}
|\lambda I-A|&=\left|\begin{matrix} \lambda-2 & 4\\ -3 & \lambda+5 \end{matrix}\right|\\
             &=(\lambda-2)(\lambda+5)-(-12)\\
             &=\lambda^2+3\lambda-10+12\\
             &=\lambda^2+3\lambda+2\\
             &=(\lambda+1)(\lambda+2)
\end{align*}
Now, we have $\lambda_1=-1$ and $\lambda_2=-2$ as eigenvalue. To find the corresponding eigenvector
\end{proof}


\begin{rmk}[Formula for eigenvalue]
For a $2\times 2$ matrix
\[\begin{bmatrix} a & b\\ c & d \end{bmatrix}\]
The possible value for the eigenvalues are given by
\[\frac{1}{2}\biggl(a+d-\sqrt{a^2+4bc-2ad+d^2}\biggl),\qquad \frac{1}{2}\biggl(a+d+\sqrt{a^2+4bc-2ad+d^2}\biggl)\]
\end{rmk}


 \begin{thebibliography}{9}

\bibitem{May}
[Demissu Gemeda]
Topics in Linear Algebra,
Addis Ababa University 2005.

\bibitem{amsshort}
[Sheldon Axler]
Linear Algebra Done Right,
Springer Publishing 1997.

\bibitem{notsoshort}
[Kolman \S  Hill]
Introduction to Linear Algebra with Applications 2000.

\end{thebibliography}

\end{document}
