\documentclass[12pt]{article}

% This first part of the file is called the PREAMBLE. It includes
% customizations and command definitions. The preamble is everything
% between \documentclass and \begin{document}.

\usepackage[margin=1in]{geometry}  % set the margins to 1in on all sides
\usepackage{graphicx}              % to include figures
\usepackage{amsmath}               % great math stuff
\usepackage{amsfonts}              % for blackboard bold, etc
\usepackage{amsthm}                % better theorem environments
\usepackage{amsthm,amssymb}


% various theorems, numbered by section

\newtheorem{thm}{Theorem}[section]
\newtheorem{lem}[thm]{Lemma}
\newtheorem{prop}[thm]{Proposition}
\newtheorem{cor}[thm]{Corollary}
\newtheorem{conj}[thm]{Conjecture}

\DeclareMathOperator{\id}{id}

\newcommand{\bd}[1]{\mathbf{#1}}  % for bolding symbols
\newcommand{\RR}{\mathbb{R}}      % for Real numbers
\newcommand{\ZZ}{\mathbb{Z}}      % for Integers
\newcommand{\col}[1]{\left[\begin{matrix} #1 \end{matrix} \right]}
\newcommand{\comb}[2]{\binom{#1^2 + #2^2}{#1+#2}}

\begin{document}


\nocite{}

\title{\textbf{Trace of a Matrix}}

\author{Miliyon T.}
\maketitle


\section*{}

\subsection*{Definition}

\begin{itemize}
\item \emph{\textbf{Trace}:} is the sum of the diagonals of the matrix. Let $A$ be a matrix of order $n$
$$
  A = \left[
    \begin{matrix}
      a_{11} & a_{12} &   \cdots & a_{1n} \\
      a_{21} & a_{22} &   \cdots & a_{2n} \\
      \vdots & \vdots &   \ddots & \vdots \\
      a_{n1} & a_{n2} & \cdots & a_{nn}
    \end{matrix}
    \right].
    $$
then the trace of A is given by
$$ \mbox{tr}(A)=a_{11}+a_{22}+\cdot\cdot\cdot+a_{nn}$$
Generally,
$$
 \mbox{tr}(A)=\sum_{i=1}^n a_{ii}
$$
\textmd{\textbf{Basic properties of trace}}
 \begin{enumerate}
           \item \mbox{tr}(A+B)=\mbox{tr}(A)+\mbox{tr}(B)
           \item \mbox{tr}(A+B)
           \item \mbox{tr}(AB)
 \end{enumerate}


\item \emph{\textbf{Similar Matrix}}: The matrices A and B $\in M_n(F)$ are Similar if there exists an invertible matrix P such that
$$B=P^{-1}AP$$

\end{itemize}


\begin{lem}
  Trace of a matrix is commutative
  $$
  \mbox{tr}(AB)=\mbox{tr}(BA)
  $$
\end{lem}

\begin{proof}
\begin{equation}
\mbox{tr}(AB)=\sum_{i=1}^n\sum_{k=1}^n A_{ik}B_{ki}
\end{equation}

\begin{equation}
\mbox{tr}(BA)=\sum_{i=1}^n\sum_{k=1}^n B_{ik}A_{ki}=\sum_{i=1}^n\sum_{k=1}^n A_{ki}B_{ik}
\end{equation}

Changing the index in (2) from $k$ to $i$ completes the proof.

\end{proof}

\begin{thm}[Similar trace Theorem]
  Similar matrices have the same trace.
\end{thm}

\begin{proof}
 Let A and B be Similar matrices by definition,
 $$ B=P^{-1}AP$$
\begin{align*}
\mbox{tr}(B)&=\mbox{tr}(P^{-1}AP)\\
     &=\mbox{tr}(P^{-1}PA) ~~~~~~    from~the~lemma~above\\
     &=\mbox{tr}(A)        ~~~~~~ \because ~P^{-1}P=I_n=1\\
\therefore \mbox{tr}(B)=\mbox{tr}(A)
\end{align*}
\end{proof}
\begin{cor}
If A and B are similar matrices, then $A_iB^i=B_iA^i$. Where $A^i =i^{th}$ column of A and $A_i=i^{th}$ row of A.
\end{cor}

 \begin{thebibliography}{9}

\bibitem{May}
[Demissu Gemeda]
Topics in Linear Algebra,
Addis Ababa University. 2005

\bibitem{amsshort}
[Sheldon Axler]
Linear Algebra Done Right,
Springer Publishing. 1997

\bibitem{notsoshort}
[Kolman \S  Hill]
Introduction to Linear Algebra with Applications.2000

\end{thebibliography}

\end{document}
